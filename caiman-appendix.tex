\section{Caiman Examples}

\subsection{Typed Select Sum}
\label{subsec:typed-select-sum}

Versions 1 and 2

\begin{lstlisting}
#version 0.1.0

tmln time(in: Event) -> Event { returns in }

sptl space(s: BufferSpace) -> BufferSpace { returns s }

extern(cpu) pure sum([i64; 4]) -> i64

val select_sum(v1: [i64; 4], v2: [i64; 4], v3: [i64; 4]) -> out: i64 {
	sum1 :- sum(v1)
	sum2 :- sum(v2)
	sum3 :- sum(v3)
	condition :- sum1 < 0
	result :- sum2 if condition else sum3
	returns result
}

fn select_sum_impl(
v1: [i64; 4] @ [node(val.v1)], 
v2: [i64; 4] @ [node(val.v2)], 
v3: [i64; 4] @ [node(val.v3)]
) 
-> i64 @ [node(val.out)] impls select_sum,time,space {
	let sum1 : i64 @ node(val.sum1) = sum(v1);
	let condition : bool @ node(val.condition) = sum1 < 0;
	if @ [node(val.result)] condition {
		let sum2 : i64 @ [node(val.sum2)] = sum(v2);
		sum2
	}
	else {
		let sum3 : i64 @ [node(val.sum3)] = sum(v3);
		sum3
	}
}

fn select_sum_impl_2(
v1: [i64; 4] @ [node(val.v1)], 
v2: [i64; 4] @ [node(val.v2)], 
v3: [i64; 4] @ [node(val.v3)]
) 
-> i64 @ [node(val.out)] impls select_sum,time,space {
	let sum2 : i64 @ [node(val.sum2)] = sum(v2);
	let sum3 : i64 @ [node(val.sum3)] = sum(v3);
	
	let sum1 : i64 @ node(val.sum1) = sum(v1);
	let condition : bool @ node(val.condition) = sum1 < 0;
	if @ [node(val.result)] condition {
		sum2
	}
	else {
		sum3
	}
}

pipeline main { select_sum_impl }
\end{lstlisting}

\subsection{Typed Select Sum GPU}
\label{subsec:typed-select-sum-gpu}

\begin{lstlisting}
#version 0.1.0

sptl space(s: BufferSpace) -> BufferSpace { returns s }

feq sum {
	extern(cpu) pure sum_cpu(x : [i32; 4]) -> out: i32
	extern(gpu) sum_gpu(x : [i32; 4]) -> out: i32
	{
		path : "gpu_sum.comp",
		entry : "main",
		dimensions : 3,
		resource {
			group : 0,
			binding : 0,
			input : x
		},
		resource {
			group : 0,
			binding : 1,
			output : out
		}
	}
}

val select_sum(v1: [i32; 4], v2: [i32; 4], v3: [i32; 4]) -> out: i32 {
	sum1 :- sum(v1)
	sum2 :- sum(v2)
	sum3 :- sum(v3)
	condition :- sum1 < 0
	result :- sum2 if condition else sum3
	returns result
}

tmln single_sync(in: Event) -> out: Event { 
	local, remote :- encode_event(in)
	sub :- submit_event(remote)
	sync :- sync_event(local, sub)
	returns sync
}

fn select_sum_impl(
v1: [i32; 4] @ [node(val.v1)], 
v2: [i32; 4] @ [node(val.v2), node(tmln.in)], 
v3: [i32; 4] @ [node(val.v3), node(tmln.in)]
) 
-> i32 impls select_sum,single_sync,space {
	let sum1 @ node(val.sum1) = sum_cpu(v1);
	let condition = sum1 < 0;
	if condition {
		let encoder = encode-begin @ node(tmln.(local, remote)) { v2_gpu, v2_gpu_sum } gpu;
		encode encoder.copy[v2_gpu @ [node(val.v2), node(tmln.remote)] <- v2];
		encode encoder.call[v2_gpu_sum <- sum_gpu'<1, 1, 1>(v2_gpu)];
		let fence = submit @ node(tmln.sub) encoder;
		let result = await @ node(tmln.sync) fence;
		
		let v2_sum = result.v2_gpu_sum;
		v2_sum
	}
	else {
		let encoder = encode-begin @ node(tmln.(local, remote)) { v3_gpu, v3_gpu_sum } gpu;
		encode encoder.copy[v3_gpu @ [node(val.v3), node(tmln.remote)] <- v3];
		encode encoder.call[v3_gpu_sum <- sum_gpu'<1, 1, 1>(v3_gpu)];
		let fence = submit @ node(tmln.sub) encoder;
		let result = await @ node(tmln.sync) fence;
		
		let v3_sum = result.v3_gpu_sum;
		@out { input: node(tmln.out), output: node(tmln.out) };
		v3_sum
	}
}

pipeline main { select_sum_impl }
\end{lstlisting}

\subsection{Caiman IR Examples}
\label{subsec:caiman-ir-examples}

First, we show an example of straightforward series of external operations:

\begin{lstlisting}
version 0.0.2

// implements:
// fn foo() -> i64 {
//     x = 3;
//     y = 2;
//     n1 = x + y;
//     n2 = x + n1;
//     return n1 + n2;
// }

ffi i64;
event %event0;
buffer_space %buffspace;
native_value %i64 : i64;

function @add(%i64, %i64) -> %i64;
function @main() -> %i64;

external-cpu-pure[impl @add] %add(i64, i64) -> i64;

value[impl default @main] %foo() -> [%out: %i64] {
    %res_t = call @add(%n1, %n2); // 8 + 5 = 13
    %res = extract %res_t 0;
    %n2_t = call @add(%x, %n1); // 3 + 5 = 8
    %n2 = extract %n2_t 0;
    %n1_t = call @add(%x, %y); // 3 + 2 = 5
    %n1 = extract %n1_t 0;
    %x = constant %i64 3;
    %y = constant %i64 2;
    return %res;
}

timeline %time(%e : %event0) -> %event0 {
    return %e;
}

spatial %space(%bs : %buffspace) -> %buffspace {
    return %bs;
}

schedule[value val = %foo, timeline time = %time, spatial space = %space]
%foo_main<time-usable, time-usable>() ->
[%out : val.%out-usable %i64] {
    %x_loc = alloc-temporary local [] i64;
    %y_loc = alloc-temporary local [] i64;
    %n1_loc = alloc-temporary local [] i64;
    %n2_loc = alloc-temporary local [] i64;
    %res_loc = alloc-temporary local [] i64;

    local-do-builtin val.%x() -> %x_loc;
    local-do-builtin val.%y() -> %y_loc;

    %x = read-ref i64 %x_loc;
    %y = read-ref i64 %y_loc;
    local-do-external %add val.%n1_t(%x, %y) -> %n1_loc;
    %n1 = read-ref i64 %n1_loc;
    local-do-external %add val.%n2_t(%x, %n1) -> %n2_loc;
    %n2 = read-ref i64 %n2_loc;
    local-do-external %add val.%res_t(%n1, %n2) -> %res_loc;
    
    %res_val = read-ref i64 %res_loc;
    return %res_val;
}

pipeline "main" = %foo_main;
\end{lstlisting}

Second, we show an example involving a call to the GPU:

\begin{lstlisting}
version 0.0.2

// Performs a single computation on the GPU,
// encoding, submitting, and waiting all in one funclet.

ffi i32;
native_value %i32 : i32;
ref %i32l : i32-local<flags=[map_read, map_write, 
  copy_src, copy_dst, storage]>;
ref %i32g : i32-gpu<flags=[map_read, map_write, 
  copy_src, copy_dst, storage]>;
event %event0;
buffer %buffer_gpu : gpu<flags = [map_read, map_write, 
  copy_src, copy_dst, storage], alignment_bits = 0, byte_size = 1024>;
buffer_space %buff_space;

function @simple(%i32) -> %i32;
function @foo(%i32) -> %i32;

external-gpu[impl @simple] %simple(%x : i32) -> [%out : i32]
{
    path : "gpu_external.comp",
    entry : "main",
    dimensionality : 3,
    resource {
        group : 0,
        binding : 0,
        input : %x
    },
    resource {
        group : 0,
        binding : 1,
        output : %out
    }
}

value[impl @foo] %foo(%x : %i32) -> %i32 {
    %c = constant %i32 1;
    %y_t = call @simple(%c, %c, %c, %x);
    %y = extract %y_t 0;
    return %y;
}

timeline %foo_time(%e : %event0) -> [%out: %event0] {
    %enc = encoding-event %e [];
    %enc1 = extract %enc 0;
    %enc2 = extract %enc 1;
    %sub = submission-event %enc2;
    %snc = synchronization-event %enc1 %sub;
    return %snc;
}

spatial %foo_space(%bs : %buff_space) -> %buff_space {
    return %bs;
}

schedule[value val = %foo, 
  timeline time = %foo_time, spatial space = %foo_space]
%foo_main<time.%e-usable, time.%out-usable>
(%x_loc : val.%x-usable %i32l)
-> [%out : val.%y-usable %i32] {
    %c_loc = alloc-temporary local [storage] i32;
    %x_gpu = alloc-temporary gpu [storage, copy_dst] i32;
    %y_gpu = alloc-temporary gpu [storage, map_read] i32;
    %y_loc = alloc-temporary local [map_write] i32;

    local-do-builtin val.%c() -> %c_loc;
    %enc = begin-encoding gpu time.%enc [%x_gpu, %y_gpu] [];
    encode-copy %enc %x_loc -> %x_gpu;
    %c = read-ref i32 %c_loc;
    encode-do %enc %simple val.%y_t(%c, %c, %c, %x_gpu) -> %y_gpu;

    %fnc = submit %enc time.%sub;
    sync-fence %fnc time.%snc;
    
    local-copy %y_gpu -> %y_loc;
    %result = read-ref i32 %y_loc;
    return %result;
}

pipeline "main" = %foo_main;
\end{lstlisting}

which includes associated WGSL code:

\begin{lstlisting}
#version 450

layout(set = 0, binding = 0) readonly buffer Input_0 {
    int field_0;
} input_0;

layout(set = 0, binding = 1) buffer Output_0 {
    int field_0;
} output_0;

layout(local_size_x = 1, local_size_y = 1, local_size_z = 1) in;
void main()
{
    output_0.field_0 = input_0.field_0 + 1;
}
\end{lstlisting}

\subsection{Caiman IR Explicated Implementation}
\label{subsec:worked-explicated-example}
Partially example in the current working version of Caiman.  Note that this example is not ``maximally" explicated, but will hopefully illustrate what can be written as--is.

\begin{lstlisting}
version 0.0.2

ffi i64;
ffi array<i64, 4>;
ref %i64l : i64-local<flags=[]>;
event %event0;
buffer_space %buffspace;
native_value %array4 : array<i64, 4>;
native_value %i64 : i64;

function @sum(%array4) -> %i64;
function @is_negative(%i64) -> %i64;
function @select_sum(%array4) -> %i64;

external-cpu-pure[impl @sum] %sum(array<i64, 4>) -> i64;
external-cpu-pure[impl @is_negative] %is_negative(i64) -> i64;

value[impl default @select_sum] %main(
  %v1 : %array4, %v2 : %array4, %v3 : %array4) -> [%out : %i64] {
    %res = select %sel %left %right;
    return %res;

    %s_t = call @sum(%v1);
    %s = extract %s_t 0;
    %sel_t = call @is_negative(%s);
    %sel = extract %sel_t 0;

    %left_t = call @sum(%v2);
    %left = extract %left_t 0;
    %right_t = call @sum(%v3);
    %right = extract %right_t 0;
}

timeline %time(%e : %event0) -> %event0 {
    return %e;
}

spatial %space(%bs : %buffspace) -> %buffspace {
    return %bs;
}

schedule[value val = %main, 
  timeline time = %time, spatial space = %space]
%select_sum_head<time-usable, time-usable>(
    %v1 : val.%v1-usable time-usable space-usable %array4,
    %v2 : val.%v2-usable time-usable space-usable %array4,
    %v3 : val.%v3-usable time-usable space-usable %array4
) -> 
    val.%out-usable time-usable space-usable %i64 
{
    %_ = alloc-temporary local [] i64;

    local-do-external %sum ? ? -> ?;
    %_ = read-ref i64 ?;

    local-do-external %is_negative ? ? -> ?;
    %sel = read-ref i64 ?;

    %djoin = default-join;
    %join = inline-join %select_sum_join [] %djoin;

    schedule-select %sel 
        [%select_sum_left, %select_sum_right] 
        [val.%res, time, space] 
        (%v2, %v3) 
        %join;
}

schedule[value val = %main, 
  timeline time = %time, spatial space = %space]
%select_sum_left<time-usable, time-usable>(
    %v2 : phi-val.%v2-usable time-usable space-usable %array4,
    %v3 : phi-val.%v3-usable time-usable space-usable %array4
) -> 
    val.%left-usable time-usable space-usable %i64
{
    %_ = alloc-temporary local [] i64;

    local-do-external %sum ? ? -> ?;
    %_ = read-ref ? ?;
    return ?;
}

schedule[value val = %main, 
  timeline time = %time, spatial space = %space]
%select_sum_right<time-usable, time-usable>(
    %v2 : phi-val.%v2-usable time-usable space-usable %array4,
    %v3 : phi-val.%v3-usable time-usable space-usable %array4
) -> 
    val.%right-usable time-usable space-usable %i64
{
    %_ = alloc-temporary local [] i64;

    local-do-external %sum val.%right_t ? -> ?;
    %_ = read-ref ? ?;
    return ?;
}

schedule[value val = %main, 
  timeline time = %time, spatial space = %space]
%select_sum_join<time-usable, time-usable>(
    %res : val.%res-usable time-usable space-usable %i64
) -> 
    val.%out-usable time-usable space-usable %i64
{
    return ?;
}

pipeline "main" = %select_sum_head;
\end{lstlisting}

\subsection{Caiman Frontend Examples}
\label{subsec:caiman-frontend-examples}

The following examples all compile and run as expected in the current build of Caiman.

Nested conditional logic:

\begin{lstlisting}
#version 0.1.0

tmln time(e: Event) -> Event { returns e }
sptl space(bs: BufferSpace) -> BufferSpace { returns bs }


val main() -> i64 {
    b :- true
    c :- false
    d :- false
    one :- 1
    two :- 2
    three :- 3
    four :- 4
    left :- one if b else two
    right :- three if c else four
    z :- left if d else right
    returns z
}

fn foo() -> i64
    impls main, time, space
{
    let d = false;
    var v;
    if d {
        let b = true;
        let two = 2;
        v = two;
        if b {
            let one = 1;
            v = one;
        }
    } else {
        let c = false;
        let four = 4;
        v = four;
        if c {
            let three = 3;
            v = three;
        }
    }
    v
}

pipeline main { foo }
\end{lstlisting}

Recursion:

\begin{lstlisting}
#version 0.1.0

tmln time(e: Event) -> Event { returns e }
sptl space(s: BufferSpace) -> BufferSpace { returns s }

val gcd(a: i64, b: i64) -> i64 {
    returns a if b == 0
            else gcd(b, a % b)
}

fn gcd_impl(a: i64, b: i64) -> i64 impls gcd, time, space {
    if b == 0 {
        a
    } else {
        gcd_impl(b, a % b)
    }
}

pipeline main { gcd_impl }
\end{lstlisting}

WGSL Function Calls:

\begin{lstlisting}
#version 0.1.0

extern(cpu) pure baz() -> i32
extern(cpu) pure bar() -> i32
extern(gpu) gpu_merge(x : i32, y: i32) -> out: i32
{
    path : "gpu_merge.comp",
    entry : "main",
    dimensions : 3,
    resource {
        group : 0,
        binding : 0,
        input : x
    },
    resource {
        group : 0,
        binding : 1,
        input : y
    },
    resource {
        group : 0,
        binding : 2,
        output : out
    }
}

val foo(c: bool) -> out: i32 {
    a :- baz()
    b :- bar()

    snd :- a if c else b

    r :- gpu_merge'<1, 1, 1>(a, snd)
    returns r
}

tmln foo_time(e: Event) -> out: Event {
    loc, rem :- encode_event(e)
    sub :- submit_event(rem)
    snc :- sync_event(loc, sub)
    returns snc
}

sptl foo_space(bs: BufferSpace) -> BufferSpace {
    returns bs
}

fn foo_impl(c: bool) -> i32 impls foo, foo_time, foo_space {
    @in {input: input(tmln.e), output: node(tmln.out) };
    let a = baz();
    let b = bar();
    let e = encode-begin @ node(tmln.(loc, rem)) 
      { a_gpu, b_gpu, y_gpu } gpu;
    encode e.copy[a_gpu <- a];
    if c {
        @in { input: node(tmln.loc), output: node(tmln.loc),
            a: node(tmln.loc), b: node(tmln.loc), e: node(tmln.rem),
            a_gpu: node(tmln.rem), b_gpu: node(tmln.rem),
              y_gpu: node(tmln.rem) };
        encode e.copy[b_gpu <- a];
    } else {
        @in { input: node(tmln.loc), output: node(tmln.loc),
            a: node(tmln.loc), b: node(tmln.loc), e: node(tmln.rem),
            a_gpu: node(tmln.rem), b_gpu: node(tmln.rem), 
              y_gpu: node(tmln.rem) };
        encode e.copy[b_gpu <- b];
    }
    @in { input: node(tmln.loc), output: node(tmln.out),
        e: node(tmln.rem),
        a_gpu: node(tmln.rem), b_gpu: node(tmln.rem), 
          y_gpu: node(tmln.rem) };
    encode e.call[y_gpu <- gpu_merge'<1, 1, 1>(a_gpu, b_gpu)];
    let f = submit @ node(tmln.sub) e;
    let r = await @ node(tmln.snc) f;
    @out { input: node(tmln.out), output: node(tmln.out) };
    r.y_gpu
}

pipeline main { foo_impl }
\end{lstlisting}

Fixed--size arrays for select sum:

\begin{lstlisting}
#version 0.1.0

feq sum {
    extern(cpu) pure sum1([i64; 4]) -> i64
    extern(cpu) pure sum2([i64; 4]) -> i64
}

val select_sum(v1: [i64; 4], v2: [i64; 4], v3: [i64; 4]) -> out: i64 {
    sum1 :- sum(v1)
    sum2 :- sum(v2)
    sum3 :- sum(v3)
    condition :- sum1 < 0
    result :- sum2 if condition else sum3
    returns result
}

sptl space(s: BufferSpace) -> BufferSpace { returns s }

tmln time(e: Event) -> Event { returns e }

fn select_sum_impl(
    v1: [i64; 4], 
    v2: [i64; 4], 
    v3: [i64; 4]
) 
-> i64 impls select_sum,time,space {
    if sum1(v1) < 0 {
        sum2(v2)
    }
    else {
        sum2(v3)
    }
}

pipeline main { select_sum_impl }
\end{lstlisting}