\subsection{\lang's Syntax}
\semanticsdocfalse

\subsubsection{Type System}

\begin{gather*}
c \in\text{constants}\\
x \in\text{variables}\\
p \in\text{primitives}\\
t \in\text{types} \\
\tau \defas \textrm{unit} \,|\, \top_p \,|\, \bot_p \,|\, t \\
f \in \text{function names} \\ 
\ifsemanticsdoc
    Only binary functions are modeled in this document, however their treatment can be easily generalized to functions with an arbitrary number of inputs.
\fi
e \defas x \,|\, c \,|\, f(e_1, e_2) \,|\, x\ \textrm{as!}\ \tau \,|\, x\ \textrm{in}\ \tau\\
C \defas \tau \, x = e \,|\, e \\
P  =  C;P \,|\, \epsilon 
\end{gather*}

\subsection{\lang's static rules}

  \begin{mathpar}
	\inferrule
	{\tau_1\leq\tau_2\qquad\Gamma\vdash e:\tau_1}
	{\Gamma\vdash e:\tau_2}
	
	\inferrule
	{\textrm{X}(c)=p}
	{\Gamma\vdash c:\bot_p}
	
	\inferrule
	{\Gamma(v)=\tau}
	{\Gamma\vdash x :\tau}
	
	\inferrule
	{\Gamma\vdash e : \tau\qquad\Gamma(v)=\tau}
	{\Gamma\vdash \tau\ x = e:\textrm{unit}}
	
	\inferrule
	{\Gamma\vdash C : \tau_1\qquad\Gamma\vdash P : \tau_2}
	{\Gamma\vdash C;P:\textrm{unit}}
	
	\inferrule
	{ }
	{\Gamma\vdash \epsilon:\textrm{unit}}
	
	\inferrule
	{\Gamma\vdash e: \top_p\qquad\tau\leq\top_p}
	{\Gamma\vdash e\ \textrm{as!}\ \tau:\tau}
	
	\inferrule
	{\Gamma\vdash e : \tau_1\qquad\textrm{P}(\tau_1,\tau_2)=f}
	{\Gamma\vdash e\ \textrm{in}\ \tau_2:\tau_2}
	
	\inferrule
	{\Gamma\vdash e_1:\tau_1\qquad\Gamma,\vdash e_2:\tau_2 \qquad \Phi(f,\tau_1,\tau_2)=\tau_3}
	{\Gamma\vdash f(e_1,e_2):\tau_3}
\end{mathpar}
\label{fig:semantics}


\subsection{Ordering rules}
\begin{mathpar}
\inferrule
{ }
{\tau\leq\tau}

\inferrule
{\tau_1 \leq \tau_2 \and \tau_2 \leq \tau_3}
{\tau_1 \leq \tau_3}

\inferrule
{ }
{\tau\leq\mathsf{unit}}

\inferrule
{ }
{\bot_p\leq\top_p}

\inferrule
{\tau\leq\top_p}
{\bot_p\leq\tau}
\end{mathpar}
\label{fig:subtype}

\subsection{Target language grammar}
Hatchling is an abstraction over sound imperative languages.

It has an \textit{operation context} $\Xi$ that maps operator names and types to their implementation. $\Xi : (\textrm{operator names}), \tau_1, \tau_2 \rightarrow \textrm{type}$.

We use the symbol $\tvdash$ for judgment in the target language

\subsubsection{Hatchling's syntax}
\begin{gather*}
c \in\textrm{constants}\\
x \in\textrm{variables}\\
p\in\textrm{primitives}\\
t\in\textrm{types}\\
\tau ::= \textrm{unit} \,|\, \top_p \,|\, \bot_p \\
o \in \textrm{operation names} \\
e ::= x \,|\, c \,|\, o(e_1, e_2) \\
c ::= \tau \, x = e\,|\, e \\
P  =  c;P \,|\, \epsilon 
\end{gather*}

\subsubsection{Hatchling's static semantics}
Subsumption
\begin{mathpar}
\infer[\subsumption]{\Gamma \tvdash e:\textrm{unit}, \Gamma}{\Gamma \tvdash e: \tau,\Gamma}
\end{mathpar}

Constants and variable declarations

\begin{mathpar}
 \infer[UNIT]{\Gamma \tvdash ():\textrm{unit}}{}
\end{mathpar}
\begin{mathpar}
\infer[VAR]
{\Gamma\vdash v:\tau}
{\Gamma\vdash \Gamma(v)=\tau}
\end{mathpar}
\begin{mathpar}
 \infer[PRIMITIVE]{\Gamma\vdash c:\bot_\mathsf{p}}{\Gamma\vdash X(c)=p} 
\end{mathpar}
\begin{mathpar}
 \infer[DECL]{\Gamma \tvdash \tau x:=e:\textrm{unit}, x\mapsto \tau}{\Gamma\tvdash e:\tau}
\end{mathpar}

Operations

\begin{mathpar}    
\infer[OP]
{\Gamma\vdash o(e_1,e_2):\tau_3}
{\Gamma\vdash e_1:\tau_1\qquad\Gamma,\vdash e_2:\tau_2 \qquad \Gamma\vdash \Xi(o, \tau_1, \tau_2)=\tau_3} 
\end{mathpar}


\subsection{Translation}

\subsubsection{Literals}
\begin{gather*}
\source{c} \triangleq c
\end{gather*}

\subsubsection{Types}
\begin{gather*}
\source{T} \triangleq \top_p \\
\text{Where we know $\top_n$ is always defined because of the structure imposed on a well formed $\Delta$.}\\
\source{\top_p} \triangleq \top_p \\
\source{\bot_p} \triangleq \top_p \\
\source{\textrm{unit}} \triangleq \textrm{unit} \\
\end{gather*}

\subsubsection{Expressions}
\begin{align*}
\source{c}_\Gamma &\triangleq c \\
\source{x}_\Gamma &\triangleq x \\
\source{\tau\ x := e}_\Gamma &\triangleq \source{\tau}\ x := \source{e}_\Gamma \\
\source{e\ \mathrm{as!}\ \tau}_\Gamma&\triangleq \source{e}_\Gamma \\
\source{e\ \mathrm{in}\ \tau_2}_\Gamma&\triangleq \source{f(e)}_\Gamma&\text{where }&\Gamma|-e:\tau_1\text{ and }f=\mathrm{P}(e,\tau_1,\tau_2)\\
\source{f(e_1, e_2)}_\Gamma&\triangleq f'(e_1, e_2)&\text{where }&\Gamma|-e:\tau_1,\Gamma|-e:\tau_2,\text{ and }f'=\Psi(f,e_1,e_2,\tau_1,\tau_2)\\
\source{\epsilon}_\Gamma&\triangleq \epsilon\\
\source{C;P}_\Gamma&\triangleq\source{C}_\Gamma;\source{P}_\Gamma\\
\end{align*}

\subsubsection{}
Additionally, we define the translation of $\Gamma$ and $\Phi$; in the target language, we replace every $\tau$ in the range of $\Gamma$ and $\Phi$ with its translation to get $\source{\Gamma}$ and $\source{\Phi}$. This is a different translation function than the one previously discussed, but for convenience we will use the same notation $\source{\Gamma}$ and $\source{\Phi}$.\\
\begin{lemma}
\label{lem:contexttr}
\begin{mathpar}
    \infer[]{
            \source{\Gamma} \tvdash \source{\Gamma}(\source{x}_\Gamma):\source{\tau}}{
                  {\source{\Gamma \vdash \Gamma(x) : \tau}_\Gamma}
        }
\end{mathpar}
\begin{mathpar}
    \infer[]{
            \source{\Gamma} \tvdash \source{\Gamma}(\source{\Phi(f, \tau_1, \tau_2)}_\Gamma):\source{\tau}}{
                  {\source{\Gamma \vdash \Phi(f, \tau_1, \tau_2) : \tau_3}_\Gamma}
        }
\end{mathpar}
\end{lemma}

\begin{proof}
Follows from the definitions of $\source{\Gamma}$ and $\source{\Phi}$.
\end{proof}

\subsection{Proof that if source type checks then translation type checks}
We use structural induction over \lang commands to show that any translation under a valid $\Delta$ type checks.
In the process of the induction we use the theorem that if a \lang expression type checks then its translation does too. This is also shown using structural induction over all \lang expressions.
\begin{theorem}
    Given any \lang expression $e$ that type checks under some $\Gamma$ to produce $\tau$, we prove that its translation $\source{e}$ also type checks under $\source{\Gamma}$ to produce $\source{\tau}$.
\end{theorem}
\begin{theorem}
	Given any \lang program $P$ that type checks under some $\Gamma$ to produce $\tau$, we prove that its translation $\source{e}$ also type checks under $\source{\Gamma}$ to produce $\source{\tau}$.
\end{theorem}
Below, I.H. = Inductive Hypothesis. Sometimes the inductive hypothesis is used over a general $\tau$. This implicitly excludes the special case of the polymorphic type.\\
We use the inversion of the type checking rules in \lang; if a statement in \lang type checks then we know that the premise of the corresponding type checking rule is true. Using this rule will be marked as T.C. \\

We split the proof into cases, one for each of the typing judgement rules in \lang.

\subsubsection{Expressions Type Check}

We present the most interesting case, for primitives. The other cases follow vry similarly.

\paragraph{Primitive rule}
\begin{mathpar}
    \inferrule*[Right=\sub]{
    \inferrule*[right=PRIM]{ }{\source{\Gamma}\tvdash c:\top_p,\source{\Gamma}} \\
    \source{\top_p} \triangleq \top_p \\
    \source{c}\triangleq c }
    {\source{\Gamma}\tvdash\source{c}:\source{\top_p}}
\end{mathpar}

Functions type check because of the constraints on a well formed $\Psi$.

\begin{mathpar}
    \inferrule*[Right=\sub]{
    \inferrule*[]{\Gamma\vdash f(e_1 : \tau_1, e_2 : \tau_2):\tau_3 }{\Phi(f, \tau_1, \tau_2) = \tau_3} \\
    \source{\Gamma}\tvdash\Psi(f, e_1, e_2, \tau_1, \tau_2) = \source{\Phi(f, \tau_1, \tau_2)}}
    {\source{\Gamma}\tvdash\source{f(e_1 : \tau_1, e_2 : \tau_2)}:\source{\tau_3}}
\end{mathpar}


\subsubsection{Commands Type Check}

\subsubsection{Declaration}
One case is presented here, and other follows suit along the same lines.
\paragraph{Declaration}
\small{
\begin{mathpar}
%\hspace*{-1.2cm}
\infer[\sub]{  %leaving this as \infer to force inlining, which doesn't seem to happen with \inferrule
    \source{\Gamma} \tvdash \source{\tau x := e}:\source{\textrm{unit}}, \source{\Gamma, x\mapsto\tau}}{
    {\inferrule*[Right=\sub]{
        \inferrule*[Right=\decl]{
            \inferrule*[Right=\IH]{
                    \inferrule*[Right=\TC]{
                        \source{\Gamma\vdash\tau x:=e:\textrm{unit},x\mapsto\tau}
                    }
                    {\source{\Gamma \vdash e : \tau}}
                }
                {\source{\Gamma} \tvdash \source{e}:\source{\tau}}
            } 
            {\source{\Gamma} \tvdash \source{\tau} \source{x} := \source{e} :\textrm{unit}, \source{x}\mapsto\source{\tau}} \\
        \source{\tau x := e} \triangleq ...
        }
        {\source{\Gamma} \tvdash \source{\tau x := e}:\textrm{unit}, \source{\Gamma}, \source{x}\mapsto\source{\tau}}} 
    \source{\Gamma, x\mapsto \tau} \triangleq 
    \source{\Gamma}, \source{x}\mapsto \source{\tau} 
    \source{\textrm{unit}} \triangleq \textrm{unit}
    }
\end{mathpar}
}