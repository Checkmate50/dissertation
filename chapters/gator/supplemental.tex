\newcount\draft\draft=0 % set to 0 for submission or publication
\documentclass[acmsmall,review,anonymous]{acmart}
\settopmatter{printfolios=true,printccs=false,printacmref=false}

% Remove some ACM template cruft for this draft/submission version.
\settopmatter{printfolios=true,printccs=false,printacmref=false}
\setcopyright{none}

\usepackage[utf8]{inputenc}
\usepackage{amsmath,amsthm,bbm,amsfonts,syntax,graphicx,tikz}
\usetikzlibrary{shapes,arrows, chains}
\usepackage[ligature, inference]{semantic}
\usepackage[T1]{fontenc}
\usepackage{mathpartir}
\usepackage{nccmath}
\usepackage{xspace}
\usepackage{xxx}
\usepackage{proof}
\usepackage{listings}
\lstset{escapeinside={<@}{@>}} % for color _coding_
\usepackage{booktabs} % for professional tables
\usepackage{subcaption}
\usepackage{bm}

\newcommand{\lang}{Gator\xspace}

\title{Geometry Types for Graphics Programming:
    \\ Supplementary Material}

\author{Dietrich Geisler}
\email{dag368@cornell.edu}
\author{Irene Yoon}
\email{ey222@cornell.edu}
\author{Horace He}
\email{hh498@cornell.edu}
\author{Aditi Kabra}
\email{ank55@cornell.edu}
\author{Yinnon Sanders}
\email{yys4@cornell.edu}
\author{Adrian Sampson}
\email{asampson@cs.cornell.edu}
\affiliation{%
  \institution{Cornell University}
  \city{Ithaca}
  \state{New York}
  \postcode{14853}
}
\date{}

% ACM publication cruft.
\acmJournal{PACMPL}
\acmVolume{1}
\acmNumber{OOPSLA}
\acmArticle{1}
\acmYear{2020}
\acmMonth{1}
\acmDOI{} % \acmDOI{10.1145/nnnnnnn.nnnnnnn}
\startPage{1}

% In draft mode, include git revision.
\ifnum\draft=1
\InputIfFileExists{revision}{}{\newcommand{\Revision}{}}
\subtitle{Draft \Revision}
\fancyfoot[LO,RE]{
  \footnotesize%
  Draft \Revision%
}
\fi

% Bibliography and citations.
\bibliographystyle{ACM-Reference-Format}
\citestyle{acmnumeric}

\newcommand{\mat}{\mathsf{mat}_{n_1{\times}n_2}}
\newcommand{\env}[1]{#1,\sigma}
\newcommand{\defas}{\mathrel{::=}}
\newenvironment{leftalign}%
{\fleqn[5pt]\csname align*\endcsname}%
{\csname endalign*\endcsname\endfleqn}
\newcommand{\alt}{\:|\:}
\newcommand{\bmark}{\textsf}

\mathlig{->}{\rightarrow}
\mathlig{|-}{\vdash}
\mathlig{=>}{\Rightarrow}

\tikzstyle{block} = [rectangle, draw, fill=gray!20,
text width=4em, text centered, rounded corners, minimum height=4em]
\tikzstyle{line} = [draw, -latex']

% Code listings configuration.
\lstdefinelanguage{GLSL}{
    keywords={vec4, vec3, vec2, mat3, mat4, void, float, int,
      if, abs, \#ifdef, \#else, \#endif, in, out, uniform,
      varying, attribute, return, void, space, is, coord, canon, type,
      declare, vec, mat, as, has, frame, object, dimension, coord},
    comment=[l]{//},
}
\lstset{
    language=GLSL,
    columns=fullflexible,
    keepspaces=true,
    showspaces=false,
    showstringspaces=false,
	keywordstyle=\bfseries,
	basicstyle=\ttfamily\small,
	basewidth=0.53em,
	aboveskip=0.9\medskipamount,
	belowskip=0.6\medskipamount,
	mathescape=true,
    lineskip={-2pt},
	% xleftmargin=\parindent,
}

% Shorthand for inline code.
\newcommand{\code}{\lstinline[keywordstyle=]}

% color definitions
\definecolor{darkolivegreen}{rgb}{0.33, 0.42, 0.18}

% Semantics declarations for the formalism.
\DeclareRobustCommand\tvdash{\mathrel{||}\joinrel\mkern-.5mu\mathrel{-}}
\newcommand{\source}[1]{\llbracket #1 \rrbracket}
\newcommand{\ld}[1]{\leq_\Delta}
\newcommand{\gen}[1]{\langle #1 \rangle} %generic
\newcommand{\sca}[0]{\textrm{scalar}}
\newcommand{\subsumption}{\textrm{SUBSUMPTION}}
\newcommand{\unit}{\textrm{UNIT}}
\newcommand{\scalar}{\textrm{SCALAR}}
\newcommand{\vect}{\textrm{VECTOR}} %both vec and vector lead to funky behavior
\newcommand{\matr}{\textrm{MATRIX}} %can't call this matrix because amsmath
\newcommand{\decl}{\textrm{DECL}}
\newcommand{\assign}{\textrm{ASSIGN}}
\newcommand{\add}{\textrm{ADDITION}}
\newcommand{\smul}{\textrm{SCAL MUL}}
\newcommand{\vmul}{\textrm{VEC MUL}}
\newcommand{\mmul}{\textrm{MAT MUL}}
\newcommand{\vcmul}{\textrm{VEC CMUL}}
\newcommand{\mcmul}{\textrm{MAT CMUL}}

\newcommand{\IH}{\textrm{I.H.}}
\newcommand{\TC}{\textrm{T.C.}}
\newcommand{\sub}{\textrm{SUBST}}

\newif\ifsemanticsdoc

\newtheorem{lemma}{Lemma}
\newtheorem{theorem}{Theorem}


\begin{document}
\renewcommand\thesection{\Alph{section}}
\setcounter{lemma}{0}
\maketitle

\section{GLSL Phong Source Code}
\label{app:llphong}

This section lists the full code for the Phong lighting model in plain GLSL~\cite{glsl}.

\begin{verbatim}
precision mediump float;

// External Function Declarations
uniform mat4 uModel;
uniform mat4 uView;
varying vec3 vNormal;
uniform vec3 uLight;
varying vec3 vPosition;

void main() {
  vec3 ambient = vec3(.1, 0., 0.);
  vec3 lightColor = vec3(0.4, 0.3, 0.9);
  vec3 specColor = vec3(1., 1., 1.);

  vec4 homWorldPos = uModel*vec4(vPosition, 1.0);
  vec3 camPos = normalize(vec3(uView*homWorldPos));
  vec3 worldNorm = 
	  normalize(vec3(uModel*vec4(vNormal, 0.0)));

  vec3 lightDir = 
    normalize(uLight - vec3(homWorldPos));
  vec3 reflectDir = reflect(lightDir, worldNorm);

  vec3 diffuse = 
    max(lightWorldDot, 0.0) * lightColor;

  float spec = pow(max(-dot(
    camPos, reflectDir), 0.), 32.);
  vec3 specular = spec * specColor;

  gl_FragColor = vec4(ambient+diffuse+specular, 1.0);
}
\end{verbatim}

\section{\lang Phong Source Code}
\label{app:hlphong}

This section lists equivalent code in \lang.

\begin{verbatim}
#"precision mediump float;";
using "../glsl_defs.lgl";

// Reference Frame Declarations

frame model has dimension 3;
frame world has dimension 3;
frame camera has dimension 3;
frame light has dimension 3;

// Global Variables

varying cart3<model>.point vPosition;
canon uniform hom<model>.transformation<world> uModel;
canon uniform hom<world>.transformation<camera> uView;
varying cart3<model>.vector vNormal;
uniform cart3<light>.point uLight;
canon uniform hom<light>.transformation<world> uLightTrans;

// Shader Code

void main() {
  color ambient = [.1, 0., 0.];
  color diffColor = [0.4, 0.3, 0.9];
  color specColor = [1.0, 1.0, 1.0];

  auto worldPos = vPosition in world;
  auto camPos = worldPos in camera;
  auto worldNorm = normalize(vNormal in world);

  auto lightDir = normalize((uLight in world) - worldPos);
  auto lightWorldDot = dot(lightDir, worldNorm);
  scalar diffuse = max(lightWorldDot, 0.0);

  auto reflectDir = normalize(reflect(-lightDir, worldNorm) in camera);

  scalar specular = pow(max(dot(normalize(-camPos), reflectDir), 0.), 32.);

  vec4 gl_FragColor = 
    vec4(ambient + diffuse * diffColor + specular * specColor, 1.0);
}
\end{verbatim}

\section{Case Study Images}

\begin{figure}
	\centering
	\begin{subfigure}[b]{0.45\linewidth}
		\centering
		\includegraphics[width=\linewidth]{fig/texture.png}
		\caption{Texture.}
	\end{subfigure}
	\hfill
	\begin{subfigure}[b]{0.45\linewidth}
		\centering
		\includegraphics[width=\linewidth]{fig/reflection.png}
		\caption{Reflection.}
	\end{subfigure}
	
	\begin{subfigure}[b]{0.45\linewidth}
		\centering
		\includegraphics[width=\linewidth]{fig/shadowmap.png}
		\caption{Shadow map.}
	\end{subfigure}
	\hfill
	\begin{subfigure}[b]{0.45\linewidth}
		\centering
		\includegraphics[width=\linewidth]{fig/microfacet.png}
		\caption{Microfacet.}
	\end{subfigure}
	\caption{Example outputs from first four renderers used in our case studies.}
\end{figure}
\begin{figure}
		\begin{subfigure}[b]{0.45\linewidth}
		\centering
		\includegraphics[width=\linewidth]{fig/bunnygoodfront.png}
		\caption{Phong}
	\end{subfigure}
	\hfill
	\begin{subfigure}[b]{0.45\linewidth}
		\centering
		\includegraphics[width=\linewidth]{fig/fog.png}
		\caption{Fog}
	\end{subfigure}
		\begin{subfigure}[b]{0.45\linewidth}
		\centering
		\includegraphics[width=\linewidth]{fig/bump.png}
		\caption{Bump Map}
	\end{subfigure}
	\hfill
	\begin{subfigure}[b]{0.45\linewidth}
		\centering
		\includegraphics[width=\linewidth]{fig/spotlight.png}
		\caption{Spotlight}
	\end{subfigure}
	\caption{Example outputs from last four renderers used in our case studies.}
\end{figure}

\section{Full Formal Semantics}
\label{app:formalism}
\input{formalism}


\bibliography{venues,refs}

\end{document}
