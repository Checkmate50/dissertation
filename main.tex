\newcount\draft\draft=0 % set to 0 for submission or publication
\documentclass[phd,tocprelim]{cornell}
%
% tocprelim option must be included to put the roman numeral pages in the
% table of contents
%
% The cornellheadings option will make headings completely consistent with
% guidelines.
%
% This sample document was originally provided by Blake Jacquot, and
% fixed up by Andrew Myers.
%
%Some possible packages to include
\usepackage{graphics}
\usepackage{moreverb}
%\usepackage{subfigure}
\usepackage{epsfig}
\usepackage{hangcaption}
\usepackage{newtxtext}
\usepackage{newtxmath}
\usepackage{palatino}

% Gator cruft
\usepackage{amsmath,bbm,amsfonts,syntax,tikz}
\usepackage{url}
\usepackage{graphicx,pstricks}
\usepackage{xspace}
\usepackage{xxx}
\usepackage{proof}
\usepackage{listings}

\lstset{escapeinside={<@}{@>}} % for color _coding_
\usepackage{booktabs} % for professional tables
\usepackage{subcaption}
\usepackage{bm}

\usetikzlibrary{shapes,arrows, chains}
\usepackage[ligature, inference]{semantic}
\usepackage[T1]{fontenc}
\usepackage{mathpartir}
\usepackage{nccmath}

\usepackage{subfiles}

\newcommand{\mat}{\mathsf{mat}_{n_1{\times}n_2}}
\newcommand{\env}[1]{#1,\sigma}
\newcommand{\defas}{\mathrel{::=}}
\newenvironment{leftalign}%
{\fleqn[5pt]\csname align*\endcsname}%
{\csname endalign*\endcsname\endfleqn}
\newcommand{\alt}{\:|\:}
\newcommand{\bmark}{\textsf}

\mathlig{->}{\rightarrow}
\mathlig{|-}{\vdash}
\mathlig{=>}{\Rightarrow}

% In draft mode, include git revision.
\ifnum\draft=1
\InputIfFileExists{revision}{}{\newcommand{\Revision}{}}
\subtitle{Draft \Revision}
\fancyfoot[LO,RE]{
	\footnotesize%
	Draft \Revision%
}
\fi

\tikzstyle{block} = [rectangle, draw, fill=gray!20,
text width=4em, text centered, rounded corners, minimum height=4em]
\tikzstyle{line} = [draw, -latex']

% Code listings configuration.
\lstdefinelanguage{GLSL}{
	keywords={void, float, int, with, auto,
		if, abs, \#ifdef, \#else, \#endif, in, out, uniform,
		varying, attribute, return, void, space, is, coord, canon, type,
		declare, vec, mat, as, has, frame, object, dimension, coord},
	comment=[l]{//},
}
\lstset{
	language=GLSL,
	columns=fullflexible,
	keepspaces=true,
	showspaces=false,
	showstringspaces=false,
	keywordstyle=\bfseries,
	basicstyle=\ttfamily\small,
	basewidth=0.53em,
	aboveskip=0.9\medskipamount,
	belowskip=0.6\medskipamount,
	mathescape=true,
	lineskip={-2pt},
	commentstyle=\color{gray},
	% xleftmargin=\parindent,
}

% Shorthand for inline code.
\newcommand{\code}{\lstinline[keywordstyle=]}

% color definitions
\definecolor{darkolivegreen}{rgb}{0.33, 0.42, 0.18}

% Semantics declarations for the formalism.
\DeclareRobustCommand\tvdash{\mathrel{||}\joinrel\mkern-.5mu\mathrel{-}}
\newcommand{\source}[1]{\llbracket #1 \rrbracket}
\newcommand{\ld}[1]{\leq_\Delta}
\newcommand{\gen}[1]{\langle #1 \rangle} %generic
\newcommand{\sca}[0]{\textrm{scalar}}
\newcommand{\subsumption}{\textrm{SUBSUMPTION}}
\newcommand{\unit}{\textrm{UNIT}}
\newcommand{\scalar}{\textrm{SCALAR}}
\newcommand{\vect}{\textrm{VECTOR}} %both vec and vector lead to funky behavior
\newcommand{\matr}{\textrm{MATRIX}} %can't call this matrix because amsmath
\newcommand{\decl}{\textrm{DECL}}
\newcommand{\assign}{\textrm{ASSIGN}}
\newcommand{\add}{\textrm{ADDITION}}
\newcommand{\smul}{\textrm{SCAL MUL}}
\newcommand{\vmul}{\textrm{VEC MUL}}
\newcommand{\mmul}{\textrm{MAT MUL}}
\newcommand{\vcmul}{\textrm{VEC CMUL}}
\newcommand{\mcmul}{\textrm{MAT CMUL}}

\newcommand{\IH}{\textrm{I.H.}}
\newcommand{\TC}{\textrm{T.C.}}
\newcommand{\sub}{\textrm{SUBST}}

\newif\ifsemanticsdoc

\newtheorem{lemma}{Lemma}
\newtheorem{theorem}{Theorem}

%if you're having problems with overfull boxes, you may need to increase
%the tolerance to 9999
\tolerance=9999

\bibliographystyle{plain}
%\bibliographystyle{IEEEbib}

\renewcommand{\caption}[1]{\singlespacing\hangcaption{#1}\normalspacing}
\renewcommand{\topfraction}{0.85}
\renewcommand{\textfraction}{0.1}
\renewcommand{\floatpagefraction}{0.75}

\title {Language Design for Geometry and Heterogeneous Reasoning in Graphics Programming}
\author {Dietrich Geisler}
\conferraldate {August}{2024}
\degreefield {Ph.D.}
\copyrightholder{Dietrich Geisler}
\copyrightyear{2024}

\begin{document}

\maketitle
\makecopyright

\begin{abstract}
	\xxx{TODO: update from Gator Abstract}
In domains that deal with physical space and geometry,
programmers need to track the coordinate systems that underpin a computation.
We identify a class of \emph{geometry bugs} that arise from confusing which coordinate system a vector belongs to.
These bugs are not ruled out by current languages for vector-oriented computing, are difficult to check for at run time, and can generate subtly incorrect output that can be hard to test for.

We introduce a type system and language that prevents geometry bugs by reflecting the coordinate system for each geometric object.
A value's \emph{geometry type} encodes its reference frame, the kind of geometric object (such as a point or a direction), and the computational representation (such as Cartesian or spherical coordinates).
We show how these types can rule out geometrically incorrect operations,
and we show how to use them to automatically generate correct-by-construction code to transform vectors between coordinate systems.
We implement a language for graphics programming, Gator, that checks geometry types and compiles to OpenGL's shading language, GLSL.
Using case studies, we demonstrate that Gator can raise the level of abstraction for shader programming and prevent common errors without inducing significant annotation overhead or performance cost.
\end{abstract}

\begin{biosketch}
Your biosketch goes here. Make sure it sits inside
the brackets.
\end{biosketch}

\begin{dedication}
This document is dedicated to all Cornell graduate students.
\end{dedication}

\begin{acknowledgements}
Your acknowledgements go here. Make sure it sits inside the brackets.
\end{acknowledgements}

\contentspage
\tablelistpage
\figurelistpage

\normalspacing \setcounter{page}{1} \pagenumbering{arabic}
\pagestyle{cornell} \addtolength{\parskip}{0.5\baselineskip}

\chapter{Introduction}

\section{SECTION 1}
The text for Section 1 goes here, without brackets.

\section{SECTION 2}

\subfile{chapters/gator/main}

\subsection{Subsection heading goes here}

Subsection 1 text

\subsubsection{Subsubsection 1 heading goes here}
Subsubsection 1 text

\subsubsection{Subsubsection 2 heading goes here}
Subsubsection 2 text

\section{SECTION 3}
Section 3 text. The dielectric constant at the air-metal interface
determines the resonance shift as absorption or capture occurs.

\begin{equation}
k_1=\frac{\omega }{c({1/\varepsilon_m + 1/\varepsilon_i})^{1/2}}=k_2=\frac{\omega
sin(\theta)\varepsilon_{air}^{1/2}}{c}
\end{equation}

\noindent
where $\omega$ is the frequency of the plasmon, $c$ is the speed of
light, $\varepsilon_m$ is the dielectric constant of the metal,
$\varepsilon_i$ is the dielectric constant of neighboring insulator,
and $\varepsilon_{air}$ is the dielectric constant of air.

\chapter{Gator: Geometry Types for Graphics Programming}

\section{The Black Kitten}
  One thing was certain, that the WHITE kitten had had nothing to
do with it:---it was the black kitten's fault entirely~\cite{aiw}.  For the
white kitten had been having its face washed by the old cat for
the last quarter of an hour (and bearing it pretty well,
considering); so you see that it COULDN'T have had any hand in
the mischief.

  The way Dinah washed her children's faces was this:  first she
held the poor thing down by its ear with one paw, and then with
the other paw she rubbed its face all over, the wrong way,
beginning at the nose:  and just now, as I said, she was hard at
work on the white kitten, which was lying quite still and trying
to purr---no doubt feeling that it was all meant for its good.

  But the black kitten had been finished with earlier in the
afternoon, and so, while Alice was sitting curled up in a corner
of the great arm-chair, half talking to herself and half asleep,
the kitten had been having a grand game of romps with the ball of
worsted Alice had been trying to wind up, and had been rolling it
up and down till it had all come undone again; and there it was,
spread over the hearth-rug, all knots and tangles, with the
kitten running after its own tail in the middle.

\section{The Reproach}

  `Oh, you wicked little thing!' cried Alice, catching up the
kitten, and giving it a little kiss to make it understand that it
was in disgrace.  `Really, Dinah ought to have taught you better
manners!  You OUGHT, Dinah, you know you ought!' she added,
looking reproachfully at the old cat, and speaking in as cross a
voice as she could manage---and then she scrambled back into the
arm-chair, taking the kitten and the worsted with her, and began
winding up the ball again.  But she didn't get on very fast, as
she was talking all the time, sometimes to the kitten, and
sometimes to herself.  Kitty sat very demurely on her knee,
pretending to watch the progress of the winding, and now and then
putting out one paw and gently touching the ball, as if it would
be glad to help, if it might.

  `Do you know what to-morrow is, Kitty?' Alice began.  `You'd
have guessed if you'd been up in the window with me---only Dinah
was making you tidy, so you couldn't.  I was watching the boys
getting in stick for the bonfire---and it wants plenty of
sticks, Kitty!  Only it got so cold, and it snowed so, they had
to leave off.  Never mind, Kitty, we'll go and see the bonfire
to-morrow.'  Here Alice wound two or three turns of the worsted
round the kitten's neck, just to see how it would look:  this led
to a scramble, in which the ball rolled down upon the floor, and
yards and yards of it got unwound again.

  `Do you know, I was so angry, Kitty,' Alice went on as soon as
they were comfortably settled again, `when I saw all the mischief
you had been doing, I was very nearly opening the window, and
putting you out into the snow!  And you'd have deserved it, you
little mischievous darling!  What have you got to say for
yourself?  Now don't interrupt me!' she went on, holding up one
finger.  `I'm going to tell you all your faults.  Number one:
you squeaked twice while Dinah was washing your face this
morning.  Now you can't deny it, Kitty:  I heard you!  What that
you say?' (pretending that the kitten was speaking.)  `Her paw
went into your eye?  Well, that's YOUR fault, for keeping your
eyes open---if you'd shut them tight up, it wouldn't have
happened.  Now don't make any more excuses, but listen!  Number
two:  you pulled Snowdrop away by the tail just as I had put down
the saucer of milk before her!  What, you were thirsty, were you?

\chapter{Online Verification of Commutativity}

\chapter{Caiman: DSL for Optimizing Heterogeneous Program Communication (Caiman)}

\appendix
\chapter{Chapter 1 of appendix}
Appendix chapter 1 text goes here

\bibliography{main}

\end{document}
